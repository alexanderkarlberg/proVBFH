\documentclass[12pt,a4]{article}


\usepackage{a4wide}
\usepackage{xspace}
\usepackage{amsmath}
\usepackage{amssymb}
\usepackage{booktabs}
\usepackage{graphicx}
\usepackage{url}
\usepackage[breaklinks=true]{hyperref}
%\usepackage[colorlinks=true]{hyperref}
\addtolength{\textwidth}{2cm}
\addtolength{\oddsidemargin}{-1cm}
\addtolength{\textheight}{2cm}
\addtolength{\topmargin}{-2cm}

\newcommand{\provbfh}{\texttt{proVFBH}\xspace}
\newcommand{\provbfhincl}{\texttt{proVFBH-inclusive}\xspace}
\newcommand{\hoppet}{\texttt{hoppet}\xspace}
\title{\texttt{proVBFH-inclusive v.2.0.2} manual}

\begin{document}
\maketitle

This document provides a short documentation for the \provbfhincl code.

%----------------------------------------------------------------------
\section{Installation}
To run \provbfhincl, you will need an installation of the following packages:
\begin{itemize}
\item \texttt{hoppet/struct-func-devel}:
  \url{http://hoppet.hepforge.org/}.  Note that it is specifically the
  struct-func-devel branch of \hoppet that is required.  It can be
  downloaded using:
\begin{verbatim}
 svn checkout https://svn.hepforge.org/hoppetsvn/branches/struct-func-devel/
\end{verbatim}
\item \texttt{LHAPDF}: \url{http://lhapdf.hepforge.org/}.
\end{itemize}
With these installed, \provbfhincl can be compiled by going to the base
directory and running
\begin{verbatim}
 ./configure [options]
 make
\end{verbatim}
Available options in the \texttt{configure} script can be accessed
through the \texttt{--help} or \texttt{-h} argument, as well as being
described in the \texttt{INSTALL} text file.

%----------------------------------------------------------------------
\section{Setting up a run}
To run \provbfhincl, you can simply call the executable
\texttt{provbfh\_incl} created by the installation steps above.
%
All parameters are passed as command line options, with the full list
of settings detailed in the section below, or accessible in
\texttt{src/parameters.f90}.
%
Since executing the code will produce several files, the recommended
usage is to start individual runs in dedicated subfolders.

An example setup:
\begin{verbatim}
 mkdir nlo-14tev; cd $_
 ../provbfh_incl -nlo -sqrts 14000 -iseed 2
\end{verbatim}
This will produce three files, two files \texttt{grids\_0002.dat} and
\texttt{grids\_0002.top} containing the grid and an output file
\texttt{xsct\_nlo\_seed0002.dat} containing the results of the run.
%
Different seed numbers and orders in
$\alpha_s$ can be executed in the same location, as they will result
in distinct output files.

% ----------------------------------------------------------------------
\section{Dihiggs production}
To run \provbfhincl for dihiggs production, one needs to compile the
\texttt{provbfhh\_incl} program with
\begin{verbatim}
 make provbfhh_incl
\end{verbatim}
which can then be executed exactly like the single Higgs program to
calculate the total cross section in VBF Higgs pair production.
% ----------------------------------------------------------------------
\section{Input parameters}
All accessible parameters can be specified as command line arguments.

The available options, and their default value (in {\bf bold}), are:
\begin{itemize}
\item\texttt{-lo, -nlo, -nnlo, {\bf -n3lo}}: Order in $\alpha_s$.
  
\item\texttt{-sqrts} {\bf 13000}: Center-of-mass energy in GeV.

\item\texttt{-scale-choice} {\bf 1}: Scale choice to use, with the options
  \begin{itemize}
  \item 0: Fixed scale at $\mu_0^2=m^2_h$.
  \item 1: Vector boson momentum $\mu_0^2(Q_i^2)=Q^2_i=-q_i^2$.
  \item 2: Common scale given by $\mu_0^2(Q_1, Q_2)=Q_1 Q_2$.
  \item 3: 
    $\mu_0^2(p_{t,H}) = \frac{m_H}{2} \sqrt{\left(\frac{m_H}{2}\right)^2
      + p_{t,H}^2}$ from \cite{Cacciari:2015jma}.
  \end{itemize}

\item\texttt{-xmuf} {\bf 1.0}: Factor $x_{\mu_F}$ multiplying the
  factorisation scale $\mu_F = x_{\mu_F} \mu_0$.

\item\texttt{-xmur} {\bf 1.0}: Factor $x_{\mu_R}$ multiplying the
  renormalisation scale $\mu_R = x_{\mu_R} \mu_0$.

\item\texttt{-pdf} {\bf PDF4LHC15\_nnlo\_mc}: Choice of PDF set (by name).

\item\texttt{-nmempdf} {\bf 0}: Member of the PDF.

\item\texttt{-mh} {\bf 125.}: Higgs mass.

\item\texttt{-hwidth} {\bf 0.00402964}: Higgs width.
  
\item\texttt{-mw} {\bf 80.398}: $W$ mass.

\item\texttt{-wwidth} {\bf 2.141}: $W$ width.

\item\texttt{-mz} {\bf 91.187}: $Z$ mass.

\item\texttt{-zwidth} {\bf 2.4952}: $Z$ width.

\item\texttt{-nf} {\bf 5}: number of quark flavours.
  
\item\texttt{-mt} {\bf 172.4}: top mass.

\item\texttt{-mb} {\bf 4.75}: bottom mass.
  
\item\texttt{-readingrid}: Use available grid if possible.

\item\texttt{-higgsbreitwigner}: Use a Breit-Wigner propagator for the higgs.

\item\texttt{-higgsmasswindow} {\bf 30.}: Number of width to integrate around the BW peak.
  
\item\texttt{-ncall1} {\bf 100000}: Number of calls for the initialisation of the integration grid.

\item\texttt{-itmx1} {\bf 12}: Number of iterations for the grid initialisation.
  
\item\texttt{-ncall2} {\bf 100000}: Number of calls for evaluating the integral.

\item\texttt{-itmx2} {\bf 12}: Number of iterations for the integration.

\item\texttt{-iseed} {\bf 10}: Random seed.

\item\texttt{-pdfuncert}: Compute PDF uncertainties on the fly. Significantly increases computation time (scales linearly with the number of PDF members). Also computes alphas uncertainty if this is included in the PDF.

\item\texttt{-3scaleuncert}: Compute 3 point (ie symmetric) scale variation.

\item\texttt{-7scaleuncert}: Compute 7 point scale variation.

\end{itemize}
\end{document}
