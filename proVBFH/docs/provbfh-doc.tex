\documentclass[12pt,a4]{article}
\usepackage{a4wide}
\usepackage{xspace}
\usepackage{amsmath}
\usepackage{amssymb}
\usepackage{booktabs}
\usepackage{graphicx}
\usepackage{url}
\usepackage[breaklinks=true]{hyperref}
%\usepackage[colorlinks=true]{hyperref}
\addtolength{\textwidth}{2cm}
\addtolength{\oddsidemargin}{-1cm}
\addtolength{\textheight}{2cm}
\addtolength{\topmargin}{-2cm}

\newcommand{\provbfh}{\texttt{proVBFH}\xspace}
\newcommand{\hoppet}{\texttt{hoppet}\xspace}
\title{\texttt{proVBFH v.1.2.1} manual}

\begin{document}
\maketitle

This manual provides a short documentation for the \provbfh code.

\tableofcontents

%----------------------------------------------------------------------
\section{Credits}
The \provbfh program was developed by Matteo Cacciari, Fr\'ed\'eric
Dreyer, Alexander Karlberg, Gavin Salam and Giulia Zanderighi and is
based on
\begin{itemize}
  \item M.~Cacciari, F.~A.~Dreyer, A.~Karlberg, G.~P.~Salam and G.~Zanderighi,
  %``Fully Differential Vector-Boson-Fusion Higgs Production at Next-to-Next-to-Leading Order,''
  Phys.\ Rev.\ Lett.\  {\bf 115} (2015) no.8,  082002
  % doi:10.1103/PhysRevLett.115.082002
  [arXiv:1506.02660 [hep-ph]],
\item F.~A.~Dreyer and A.~Karlberg,
  %``Vector-Boson Fusion Higgs Production at Three Loops in QCD,''
  Phys.\ Rev.\ Lett.\  {\bf 117} (2016) no.7,  072001
  % doi:10.1103/Phys\-RevLett.117.072001
  [arXiv:1606.00840 [hep-ph]].
\end{itemize}
%
The non-factorisable corrections are implemented as described in
\begin{itemize}
\item F.~A.~Dreyer, A.~Karlberg and L.~Tancredi,
  % ``On the impact of non-factaorisable corrections in VBF single and double Higgs production,''
  [arXiv:2005.11334 [hep-ph]].
\end{itemize}
%
The code itself relies heavily on the machinery of the
\texttt{POWHEG-BOX}, the implementation of VBF Hjjj production in the
\texttt{POWHEG-BOX}, and the VBF phase as implemented in the VBF
process in the \texttt{POWHEG-BOX}.
%
We therefore suggest that the following three publications be cited in
addition to the above two when the program is being used to produce
scientific results:

\begin{itemize}
\item S.~Alioli, P.~Nason, C.~Oleari and E.~Re,
  %``A general framework for implementing NLO calculations in shower Monte Carlo programs: the POWHEG BOX,''
  JHEP {\bf 1006} (2010) 043
  % doi:10.1007/JHEP06(2010)043
  [arXiv:1002.2581 [hep-ph]].
\item P.~Nason and C.~Oleari,
  %``NLO Higgs boson production via vector-boson fusion matched with shower in POWHEG,''
  JHEP {\bf 1002} (2010) 037
  % doi:10.1007/JHEP02(2010)037
  [arXiv:0911.5299 [hep-ph]].
\item B.~J\"ager, F.~Schissler and D.~Zeppenfeld,
  %``Parton-shower effects on Higgs boson production via vector-boson fusion in association with three jets,''
  JHEP {\bf 1407} (2014) 125
  % doi:10.1007/JHEP07(2014)125
  [arXiv:1405.6950 [hep-ph]].
\end{itemize}

When using results at next-to-next-to-next-to-leading order (N$^3$LO), the
original references on third order coefficient functions should be
cited as well
\begin{itemize}
\item A.~Vogt, S.~Moch and J.~A.~M.~Vermaseren,
  %``The Three-loop splitting functions in QCD: The Singlet case,''
  Nucl.\ Phys.\ B {\bf 691} (2004) 129
  % doi:10.1016/j.nuclphysb.2004.04.024
  [hep-ph/0404111].
\item S.~Moch, J.~A.~M.~Vermaseren and A.~Vogt,
  %``The Longitudinal structure function at the third order,''
  Phys.\ Lett.\ B {\bf 606} (2005) 123
  % doi:10.1016/j.physletb.2004.11.063
  [hep-ph/041.2.1].
\item J.~A.~M.~Vermaseren, A.~Vogt and S.~Moch,
  %``The Third-order QCD corrections to deep-inelastic scattering by photon exchange,''
  Nucl.\ Phys.\ B {\bf 724} (2005) 3
  % doi:10.1016/j.nuclphysb.2005.06.020
  [hep-ph/0504242].
\item S.~Moch, M.~Rogal and A.~Vogt,
  %``Differences between charged-current coefficient functions,''
  Nucl.\ Phys.\ B {\bf 790} (2008) 317
  % doi:10.1016/j.nuclphysb.2007.09.022
  [arXiv:0708.3731 [hep-ph]].
\end{itemize}

Finally, when evaluating the non-factorizable NNLO corrections, the following
paper should additionally be cited
\begin{itemize}
\item T.~Liu, K.~Melnikov and A.~A.~Penin,
  % ``Nonfactorizable QCD Effects in Higgs Boson Production via Vector Boson Fusion,''
  Phys. Rev. Lett. \textbf{123} (2019) no.12, 122002
  %doi:10.1103/PhysRevLett.123.122002
  [arXiv:1906.10899 [hep-ph]].
\end{itemize}
\subsection{Dependencies}
To run \provbfh, you will need an installation of the following packages:
\begin{itemize}
\item \texttt{hoppet}:
  \url{https://github.com/hoppet-code/hoppet}. At least v.1.3.0.
\item \texttt{LHAPDF}: \url{http://lhapdf.hepforge.org/}.
\item \texttt{FastJet}: \url{http://fastjet.fr/}.
\item \texttt{gfortran} version 4.6 or later.
\end{itemize}
%
The corresponding references for the first three of these packages
should also be cited when using \provbfh.

\section{Installation}
With the above dependencies installed, \provbfh can be compiled by
going to the base directory and running
\begin{verbatim}
 ./configure [options]
 make
\end{verbatim}
Available options in the \texttt{configure} script can be accessed
through the \texttt{--help} or \texttt{-h} argument, and are also
described in the \texttt{INSTALL} file.

Along with the main \provbfh executable, a program to combine runs,
\texttt{combine\_runs}, will be generated in the \texttt{aux} folder.

%----------------------------------------------------------------------
\section{Setting up a run}
\provbfh is partially based on the \texttt{VBF\_HJJJ} process of the
\texttt{POWHEG-BOX} and uses the same machinery to set up parallel
runs.
%
Users familiar with the \texttt{POWHEG-BOX} will therefore find
running \provbfh straightforward.
%
Given the complexity of the code it is necessary to run it in parallel
and at the end of this section we will give some recommendations on
how many parallel runs should be carried out.
%
An example configuration for running \provbfh is given in the
\texttt{example/} folder.
%
The three files necessary to run the code are
\begin{itemize}
\item \texttt{pwgseeds.dat}: contains a list of integers to be read by
  the program
\item \texttt{powheg.input}: contains all the technical parameters of
  the run. They are details below.
\item \texttt{vbfnlo.input}: contains a number of physical
  parameters. They are detailed below.
\end{itemize}
  %
From this folder, the program can be started with
\begin{verbatim}
 ../proVBFH
\end{verbatim}
%
The program will ask for an input corresponding to a line number in
the file pwgseeds.dat.

When running the program at either NLO or NNLO There are two stages
(controlled by the flag \texttt{parallelstage} in
\texttt{powheg.input}) that one has to go through in order to compute
the cross section.
%
During the first stage the program will generate adaptive grids to be
used during stage 2, where the cross section is computed.
%
It is possible to run stage 1 through several iterations.
%
The program reads the flag \texttt{xgriditeration}.
%
If this is set equal to '1' the program will run starting from uniform
grids.
%
If it is set to '$n$' the program will look for grids generated during
iteration '$n-1$' and use them as the initial grids.
%
When stage 2 is then invoked the program looks for the most recent
grids.
%
The output of stage 2 is a file with histograms called
\texttt{pwg-XXXX-(N)NLO.top} where \texttt{XXXX} is the integer which
was used to invoke the program.

At LO and N$^3$LO (inclusive) the program only runs in one stage since
the generation of grids is very fast for Born kinematics.
%
In this case the two flags described above are therefore not used by
the program.

\subsection{The Analysis}
In the event where differential distributions are requested, the
analysis will be taken from the file
\texttt{analysis/user\_analysis.f}.
%
This file can be modified to remove or add any histograms from the
output files.
%
There is an intricate relationship between the analysis used for
generating histograms and for the analysis used to impose phase space
cuts.
%
Therefore, while a user can freely add or remove histograms, one
should be careful with changing the cuts.
%
In \texttt{user\_analysis.f} there are three routines which are also
used by the phase space analysis.
%
They are \texttt{setup\_vbf\_cuts}, \texttt{buildjets} and
\texttt{vbfcuts}.

\begin{itemize}
  \item \texttt{setup\_vbf\_cuts}: This routine reads the cuts from
    \texttt{powheg.input}
  \item \texttt{buildjets}: builds an array of up to 4 jets using $R$
    from the input card and anti-$k_t$. If one wants to define jets in
    a different way this routine can be modified.
  \item \texttt{vbfcuts}: takes the jets of the above built and
    decides whether or not they will pass the VBF cuts defined in
    \texttt{setup\_vbf\_cuts}. If they are not passed the analysis
    exits. Otherwise it proceeds to fill the histograms.
\end{itemize}

If one wants to create an analysis with multiple sets of cuts (for
instance to have a set of loose/tight cuts or a central jet veto) these
cuts have to be applied \emph{after} the three routines.

\subsection{A few recommendations}
\provbfh requires a significant amount of computer power to produce
smooth distributions at NNLO.
%
Under typical VBF cuts we suggest the following technical input
parameters:

\begin{itemize}
\item \texttt{ncall1 500000}
\item \texttt{ncall2 5000000}
\item \texttt{itmx2 3}
\end{itemize}

Stage 1 should go through $3$ iterations on $200$ machines.
%
On a typical machine in $2017$ each iteration should take $2-3$ hours.
%
Stage 2 should be carried out on $2500$ machines and will take $1-2$
days depending on how complicated the analysis is and how loose the
VBF cuts are.
%
Due to the way the combination of each run is carried out it is better
to have a high number of independent runs rather than few runs with a
very high \texttt{ncall2}.

With these settings the fiducial cross section will be computed with a
permille statistical uncertainty and distributions will have percent
level uncertainties.

\subsection{Scale variations}
One can vary the renormalisation and factorisation scales used in
the program to estimate the residual theoretical uncertainties.
%
We recommend a $3$ point scale variation $\mu_R=\mu_F$ with
$\mu_{R,F}=\{1/2,1,2\}\mu_0$.
%
A seven point scale variation, where the scales are varied
independently, does not lead to significantly larger uncertainty
estimates.

\subsection{Combining runs}
We provide a script to combine independent runs.
%
The script will automatically look for outliers in each bin and remove
them.
%
It can be invoked by
%
\begin{verbatim}
../aux/combine_runs <list of files>
\end{verbatim}
%
where the list of files can be given using wildcards, eg
\texttt{*NNLO*top}.
%
This will produce a file called \texttt{total\_distrib.top} containing
the combined result.


% ----------------------------------------------------------------------
\section{Input parameters}
\subsection{\texttt{powheg.input} card}
Most parameters are changed in the \texttt{powheg.input} file.
%
Available options:
\begin{itemize}
\item \texttt{qcd\_order}: 1: LO, 2: NLO, 3:NNLO, 4:N3LO (only for
  inclusive results).

\item \texttt{inclusive\_only}: 0: Computes the cross section fully
  differentially in the jets. 1: computes only the inclusive VBF cross
  section differentially in the Higgs momentum. (default: 0)

\item\texttt{ebeam1}: energy of beam 1 in GeV.
  
\item\texttt{ebeam2}: energy of beam 2 in GeV.

\item\texttt{lhans1}: pdf set for hadron 1 (LHA numbering).

\item\texttt{lhans2}: pdf set for hadron 2 (LHA numbering).

\item\texttt{renscfact}: renormalisation scale factor: muren = muref *
  renscfact.
  
\item\texttt{facscfact}: factorisation scale factor: mufact = muref *
  facscfact.
  
\item\texttt{nonfact}: 0: compute factorisable cross-section, 1: compute
  non-factorisable coefficient. (Requires qcd\_order=1)
  
\item\texttt{higgsbreitwigner}: 0: Narrow width, 1: Breit Wigner for Higgs.
  
\item\texttt{higgsmasswindow}: How many widths we integrate around the
  Breit Wigner peak.
  
\item\texttt{runningscales}: 0: $m_h$. 1: scale of 1506.02660. 2: $Q_1$
  on the upper line and $Q_2$ on the lower line. (default: 0, i.e. no
  running scales)
                  
\item\texttt{ncall1}: number of calls for initializing the integration grid.
  
\item\texttt{ncall2}: number of calls for computing the integral.
  
\item\texttt{itmx1}: number of iterations for initializing the
  integration grid.
  
\item\texttt{itmx2}: number of iterations for computing the integral .
  
\item\texttt{xgriditeration}: identifier for grid generation.
  
\item\texttt{parallelstage}: 1: generate grids. 2: compute integral.

\item\texttt{fakevirt}: Useful for generating stage 1 grids faster as
  it uses the Born instead of the virt which is faster.

\item\texttt{phspcuts}: (1:on ; 0: off (default)) Turns on/off analysis
  cuts at the phasespace generation stage. Significantly reduces run
  time. Turning these off will make the integration unstable.
\end{itemize}

\subsubsection{Analysis cuts}
\begin{itemize}
\item\texttt{ptalljetmin}: Minimum $p_t$ for all jets.

\item\texttt{yjetmax}: Rapidity acceptance for all jets.

\item\texttt{Rsep\_jjmin}: $R$ to be used in jet algorithm.

\item\texttt{ptjetmin}: Minimum $p_t$ for the two tagging jets.

\item\texttt{mjjmin}: Minimum invariant mass for the tagging jets.

\item\texttt{deltay\_jjmin}: Rapidity separation between tagging jets.

\item\texttt{jet\_opphem}: (0/1) require the tagging jets in opposite
  detector hemispheres.
\end{itemize}

\subsubsection{Advanced settings}
Several additional options are present in the code and are required by
the \texttt{POWHEG-BOX}.
%
In principle, the user should not modify these settings as doing so
can risk producing useless results.
%
We detail the most important ones here, additional settings can be
found in the \texttt{POWHEG-BOX} documentation.
\begin{itemize}  
\item\texttt{iseed}: initialize random number sequence.
  
\item\texttt{ih1}: hadron 1 (1 for protons, -1 for antiprotons).
  
\item\texttt{ih2}: hadron 2 (1 for protons, -1 for antiprotons).

\item\texttt{maxseeds}: Maximum number of seeds that powheg can run
  with.

\item\texttt{manyseeds}: Used to perform multiple runs with different
  random seeds in the same directory.  If set to 1, the program asks
  for an integer j; The file pwgseeds.dat at line j is read, and the
  integer at line j is used to initialize the random sequence for the
  generation of the event.
                 
\end{itemize}

\subsection{\texttt{vbfnlo.input} card}
\begin{itemize}
\item\texttt{HMASS}: Higgs mass.
  
\item\texttt{TOPMASS}: Top mass.
  
\item\texttt{TAU\_MASS}: Tau mass.
  
\item\texttt{BOTTOMMASS}: Bottom Pole mass.
  
\item\texttt{CHARMMASS}: Charm Pole mass.
  
\item\texttt{FERMI\_CONST}: Fermi Constant.
  
\item\texttt{WMASS}: W mass.
  
\item\texttt{ZMASS}: Z mass.
  
\item\texttt{WWIDTH}: W width.
  
\item\texttt{ZWIDTH}: Z width.
  
\item\texttt{NFLAVOUR}: Number of quark flavours.
  
\item\texttt{HWIDTH}: Higgs width.

\end{itemize}
\end{document}
